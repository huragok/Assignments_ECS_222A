%%%%%%%%%%%%%%%%%%%%%%%%%%%%%%%%%%%%%%%%%
% Structured General Purpose Assignment
% LaTeX Template
%
% This template has been downloaded from:
% http://www.latextemplates.com
%
% Original author:
% Ted Pavlic (http://www.tedpavlic.com)
%
% Note:
% The \lipsum[#] commands throughout this template generate dummy text
% to fill the template out. These commands should all be removed when 
% writing assignment content.
%
%%%%%%%%%%%%%%%%%%%%%%%%%%%%%%%%%%%%%%%%%

%----------------------------------------------------------------------------------------
%   PACKAGES AND OTHER DOCUMENT CONFIGURATIONS
%----------------------------------------------------------------------------------------

\documentclass{article}

\usepackage{fancyhdr} % Required for custom headers
\usepackage{lastpage} % Required to determine the last page for the footer
\usepackage{extramarks} % Required for headers and footers
\usepackage{graphicx} % Required to insert images
\usepackage{lipsum} % Used for inserting dummy 'Lorem ipsum' text into the template
\usepackage{amsmath}
\usepackage{xcolor}
\usepackage{listings}
\usepackage[toc,page]{appendix}
\usepackage{algorithm}
\usepackage{algorithmic}

% Margins
\topmargin=-0.45in
\evensidemargin=0in
\oddsidemargin=0in
\textwidth=6.5in
\textheight=9.0in
\headsep=0.25in 

\linespread{1.1} % Line spacing

% Set up the header and footer
\pagestyle{fancy}
\lhead{\hmwkAuthorName} % Top left header
\chead{\hmwkClass\ (\hmwkClassInstructor): \hmwkTitle} % Top center header
\rhead{\firstxmark} % Top right header
\lfoot{\lastxmark} % Bottom left footer
\cfoot{} % Bottom center footer
\rfoot{Page\ \thepage\ of\ \pageref{LastPage}} % Bottom right footer
\renewcommand\headrulewidth{0.4pt} % Size of the header rule
\renewcommand\footrulewidth{0.4pt} % Size of the footer rule

\setlength\parindent{0pt} % Removes all indentation from paragraphs

%----------------------------------------------------------------------------------------
%   DOCUMENT STRUCTURE COMMANDS
%   Skip this unless you know what you're doing
%----------------------------------------------------------------------------------------

% Header and footer for when a page split occurs within a problem environment
\newcommand{\enterProblemHeader}[1]{
    \nobreak\extramarks{#1}{#1 continued on next page\ldots}\nobreak
    \nobreak\extramarks{#1 (continued)}{#1 continued on next page\ldots}\nobreak
}

% Header and footer for when a page split occurs between problem environments
\newcommand{\exitProblemHeader}[1]{
    \nobreak\extramarks{#1 (continued)}{#1 continued on next page\ldots}\nobreak
    \nobreak\extramarks{#1}{}\nobreak
}

\setcounter{secnumdepth}{0} % Removes default section numbers
\newcounter{homeworkProblemCounter} % Creates a counter to keep track of the number of problems
\setcounter{homeworkProblemCounter}{0}

\newcommand{\homeworkProblemName}{}
\newenvironment{homeworkProblem}[1][Problem \arabic{homeworkProblemCounter}]{ % Makes a new environment called homeworkProblem which takes 1 argument (custom name) but the default is "Problem #"
    \stepcounter{homeworkProblemCounter} % Increase counter for number of
% problems
    \renewcommand{\homeworkProblemName}{#1} % Assign \homeworkProblemName the
% name of the problem
    \section{\homeworkProblemName} % Make a section in the document with the
% custom problem count
    \enterProblemHeader{\homeworkProblemName} % Header and footer within the
% environment
}{
    \exitProblemHeader{\homeworkProblemName} % Header and footer after the
% environment
}

\newcommand{\problemAnswer}[1]{ % Defines the problem answer command with the content as the only argument
    \noindent\textbf{\emph{Answer: }}#1 % Just put a keyword Answer in
    % bold/italic at the beginning
}

\newcommand{\homeworkSectionName}{}
\newenvironment{homeworkSection}[1]{ % New environment for sections within homework problems, takes 1 argument - the name of the section
    \renewcommand{\homeworkSectionName}{#1} % Assign \homeworkSectionName to the
% name of the section from the environment argument
    \subsection{\homeworkSectionName} % Make a subsection with the custom name
% of the subsection
    \enterProblemHeader{\homeworkProblemName\ [\homeworkSectionName]} % Header
% and footer within the environment
}{
    \enterProblemHeader{\homeworkProblemName} % Header and footer after the
% environment
}

\newtheorem{theorem}{Theorem}[homeworkProblemCounter]
\newtheorem{lemma}[theorem]{Lemma}
\newtheorem{proposition}[theorem]{Proposition}
\newtheorem{corollary}[theorem]{Corollary}

\newenvironment{proof}[1][Proof]{
    \begin{trivlist}
        \item[\hskip \labelsep {\bfseries #1}]
    }{
        \end{trivlist}
}
\newenvironment{definition}[1][Definition]{
    \begin{trivlist}
        \item[\hskip \labelsep {\bfseries #1}]
    }{
        \end{trivlist}
}
\newenvironment{example}[1][Example]{
    \begin{trivlist}
        \item[\hskip \labelsep {\bfseries #1}]
    }{
        \end{trivlist}
    }
\newenvironment{remark}[1][Remark]{
    \begin{trivlist}
        \item[\hskip \labelsep {\bfseries #1}]
    }{
        \end{trivlist}
}

\newcommand{\qed}{
    \nobreak \ifvmode \relax \else
    \ifdim\lastskip<1.5em \hskip-\lastskip
    \hskip1.5em plus0em minus0.5em \fi \nobreak
    \vrule height0.75em width0.5em depth0.25em\fi
}

\lstset{
    frame=single,
    breaklines=true,
    postbreak=\raisebox{0ex}[0ex][0ex]{\ensuremath{\color{red}\hookrightarrow\space}}
}
   
%----------------------------------------------------------------------------------------
%   NAME AND CLASS SECTION
%----------------------------------------------------------------------------------------

\newcommand{\hmwkTitle}{Assignment\ \#6} % Assignment title
\newcommand{\hmwkDueDate}{Thursday,March\ 5,\ 2015} % Due date
\newcommand{\hmwkClass}{ECS\ 222A} % Course/class
\newcommand{\hmwkClassTime}{TR 4:40pm-6:00pm} % Class/lecture time
\newcommand{\hmwkClassInstructor}{Daniel Gusfield} % Teacher/lecturer
\newcommand{\hmwkAuthorName}{Wenhao Wu} % Your name

%----------------------------------------------------------------------------------------
%   TITLE PAGE
%----------------------------------------------------------------------------------------

\title{
    \vspace{2in}
    \textmd{\textbf{\hmwkClass:\ \hmwkTitle}}\\
    \normalsize\vspace{0.1in}\small{Due\ on\ \hmwkDueDate}\\
    \vspace{0.1in}\large{\textit{\hmwkClassInstructor\ \hmwkClassTime}}
    \vspace{3in}
}

\author{\textbf{\hmwkAuthorName}}
\date{} % Insert date here if you want it to appear below your name

%----------------------------------------------------------------------------------------

\begin{document}

    \maketitle
    
    %----------------------------------------------------------------------------------------
    %   TABLE OF CONTENTS
    %----------------------------------------------------------------------------------------
    
    %\setcounter{tocdepth}{1} % Uncomment this line if you don't want subsections listed in the ToC
    
    \newpage
    \tableofcontents
    \newpage

    %----------------------------------------------------------------------------------------
    %   PROBLEM 1
    %----------------------------------------------------------------------------------------
    \begin{homeworkProblem}
        We saw a randomized algorithm that tries to find a global $Min$ cut
        in an undirected, unweighted graph $G$ with $m$ edges. Now suppose we want
        to find a cut that has a \emph{large} number of edges, i.e., a partition
        of the nodes of $G$ into two sets $S$ and $T$ so that the number of
        edges that have one node in $S$ and one node in $T$ is large. Denote the
        maximum possible number as $Max(G)$. The problem of finding $Max(G)$ is
        NP-hard, so we would like a randomized algorithm that finds an $S$, $T$
        cut where the \emph{expected} number of edges that cross the cut is a
        large fraction of $Max(G)$.
        
        Here is a particularly brainless algorithm that does it. For each
        vertex, flip a coin: if the coin comes up heads, put the vertex in $S$,
        otherwise, put it in $T$. Assume that the coin is fair, i.e., the
        probability of heads is $1/2$.
        
        \begin{homeworkSection}{\homeworkProblemName(a)}
            Using this randomized algorithm, what is the expected number of
            edges that have one node in $S$ and one in $T$? Explain.
            
            \vspace{10pt}
            \problemAnswer{
                \begin{lemma}
                    \label{lemma:E_cross_u}
                    Denote the expected number of edges $(u, v)$ from vertex $u$
                    such that $u\in S$ AND $v\in T$ or $u\in T$ AND $v\in S$ as
                    $E_{cross}(u)$. Then we have
                    \begin{align}
                        E_{cross}(u) = \frac{d(u)}{2}
                    \end{align}
                    in which $d(u)$ is the degree of vertex $u$.
                \end{lemma}
                \begin{proof}
                    Firstly we have $P\{u\in S\} = P\{u\in T\} = 1/2$. The
                    probability that there among the $d(u)$ neighbors of vertex
                    $u$, $k$ of them belong to $S$ while $d(u)-k$ of belong to
                    $T$ is 
                    \begin{align}
                        P_u(|S|=k) & =
                        \left(\begin{array}{c}k\\ d(u)\end{array}\right)
                        \left(\frac{1}{2}\right) ^ k\left(\frac{1}{2}\right) ^
                        {d(u)-k} \notag\\
                        & = \left(\begin{array}{c}k\\ d(u)\end{array}\right)
                        \left(\frac{1}{2}\right) ^ {d(u)}
                    \end{align}
                    Consequently,
                    \begin{align}
                        E_{cross}(u) & = \sum_{k = 0}^{d(u)} P\{u\in
                        S\}P_u(|S|=k)(d(u) - k) + \sum_{k = 0}^{d(u)} P\{u\in
                        T\}P_u(|S|=k)k \notag \\
                        & = 2\cdot\frac{1}{2} \sum_{k = 0}^{d(u)}
                        k\left(\begin{array}{c}k\\
                        d(u)\end{array}\right) \left(\frac{1}{2}\right) ^ {d(u)}
                        \notag \\
                        & = \frac{d(u)}{2}    
                    \end{align}
                \end{proof}
                Consequently, the expected number of eges that have one node in
                $S$ and one in $T$, according to Lemma~\ref{lemma:E_cross_u} and
                the handshake lemma, is
                \begin{align}
                    E_{cross} & = \frac{1}{2}\sum_{u\in V} E_{cross}(u) \notag
                    \\
                    & = \frac{|E|}{2}
                \end{align}
                where $|E|$ is the number of edges in graph $G$.
                
            }
        \end{homeworkSection}
        
        \begin{homeworkSection}{\homeworkProblemName(b)}
            Prove that in any undirected graph $G$, there is a cut that contains
            at least half of the edges in $G$.
            
            \vspace{10pt}
            \problemAnswer{
                We note there are a total number of $2^{|V| - 1}$ possible cuts
                (taking into consideration of the symmetry between $S$ and $T$),
                and the random algorithm results in any of them with probability
                $(1/2)^{|V| - 1}$. Denote the number of edges contained in cut
                $C$ as $n_{cross}(C)$. Assume for contradiction that $n_{cross}(C)
                < |E|/2$, then we have
                \begin{align}
                    E_{cross} & = \sum_{C}\left(\frac{1}{2}\right)^{|V| -
                    1}n_{cross}(C) \notag \\
                    & < \sum_{C}\left(\frac{1}{2}\right)^{|V| -
                    1}\frac{|E|}{2} \notag \\
                    & = \frac{|E|}{2}
                \end{align}
                which is in contradiction with the solution to Problem 1(a).
                Consequently, there is a cut that contains at least half of the
                edges in $G$.
                
            }
        \end{homeworkSection}
        
        \begin{homeworkSection}{\homeworkProblemName(c)}
            Often in the analysis of social media, people build graphs
            representing who knows or likes (or hates or dates) whomever else.
            Then they analyze the graphs for particular features, such as large
            cliques, large independent sets, small cuts, nodes with high degree,
            the number of nodes of degree 1, number of triangles, etc. and
            they ascribe a ``meaning'' to each of these features. For example, a
            node with high degree is a ``hub'' or ``kingpin''; and a large
            clique represents a ``socially cohesive unit''; and a small cut is a
            ``bottleneck'', etc. This same approach to studying interactions is
            used in a huge variety of other systems. For example, graphs (call
            them ``biological networks'') are also to represent interactions
            between molecules, or between animals in a biological system, and
            biological, behavioural or chemical meanings are ascribed to
            features in these graphs. In general, such networks are called
            ``interaction networks''.
            
            Now consider a ``large cut'' as a feature of an interaction network.
            Make up (use your imagination) as many meanings you can for a large
            cut in a social or biological network, or any other interaction
            network you can describe.
            
            Given the fact stated in problem 1b, how large must a cut be before
            it could plausibly say anything meaningful about the interactions
            represented in an interaction graph? If you were a ``network
            analyst'' trying to find meaningful information from an interaction
            network, and you didn't know the fact stated in problem 1b, would
            that be a problem?
            
            \vspace{10pt}
            \problemAnswer{
                As an example of a large cut, consider a conflict graph in which
                each node represents ``a student'' and each edge represents
                ``hatred'' between two students. The teacher is having a big
                headache since the students in his class hates each other so
                much, that he decides to divide the class into 2 so that the
                total conflicts is minimized in each subclass. To do so he will
                need to find a large cut, so that a large number of conflict
                across the cut can be removed with this division.
                
                A cut must be greater than $|E|/2$ to be meaningful, since
                according to Problem 1b, there is going to be a cut of size of
                at least $|E|/2$ anyway in any graph $G$.
            
            }
        \end{homeworkSection}
    \end{homeworkProblem}
    
    %----------------------------------------------------------------------------------------
    %   PROBLEM 2
    %----------------------------------------------------------------------------------------
    \begin{homeworkProblem}
        Extend the analysis done for 3-SAT (in class and also in Section 13.4)
        to 4-SAT, i.e., the assumption that every clause has 4 literals. Then
        generalize to t-SAT for any fixed integer $t$. That is, generalize
        statments 13.14, 13.15, and 13.16, and justify your answers.
            
        \vspace{10pt}
        \problemAnswer{
            \begin{lemma}
                \label{lemma:expected_tAT}
                (Generalization of statement 13.14) Consider a $t$-SAT formula,
                where each clause has $t$ different variables. The expected
                number of clauses satisfied by a random assignment is within an
                approximation factor $1-(1/2)^t$ of optimal.
            \end{lemma}
            \begin{proof}
                Denote $Z_i = 1$ if the $i$-th clause is true and $Z_i = 0$
                otherwise. The expectation of the random variable $Z_i$ is
                \begin{align}
                    E[Z_i] & = P\{Z_i = 1\}\cdot 1 + P\{Z_i = 0\}\cdot 0
                    \notag \\
                    & = P\{Z_i = 1\} \notag \\
                    & = 1 - P\{\mbox{All $t$ literals in the $i$-th clauses
                    are false}\} \notag \\
                    & = 1 - \left(\frac{1}{2}\right)^t
                \end{align}
                Consequently, the expected number of clauses satisfied by the
                random assignment is
                \begin{align}
                    E[Z] & = \sum_{i=1}^kE[Z_i] \notag\\
                    & = \left[1 - \left(\frac{1}{2}\right)^t\right]k
                \end{align}
                where $k$ is the total number of clauses.
            \end{proof}
            
            \begin{lemma}
                \label{lemma:existence_truth_assignment}
                (Generalization of statement 13.15) For every instance of
                $t$-SAT, there is a truth assignment that satisfies at least a
                $1-(1/2)^t$ fraction of all clauses.
            \end{lemma}
            \begin{proof}
                Denote the probability that the random algorithm results in
                assignment $A_l$ as $p(A_l)$ and the number of clauses satisfied
                by $A_l$ as $Z(A_l)$. Assume for contradiction that $Z(A_l) <
                (1-(1/2)^t)k$ for all assignment $A_l$. Then we have
                \begin{align}
                    E[Z] & = \sum_l p(A_l)Z(A_l) \notag \\
                    & < \sum_l p(A_l)\left[1-
                    \left(\frac{1}{2}\right)^t\right]k \notag \\
                    & = \left[1-\left(\frac{1}{2}\right)^t\right]k
                \end{align}
                which is in contradiction with
                Lemma~\ref{lemma:expected_tAT}. Consequently,
                there is a truth assignment that satisfies at least a
                $1-(1/2)^t$ fraction of all clauses.
            \end{proof}
            
            \begin{lemma}
                \label{lemma:poly_alg}
                (Generalization of statement 13.16) There is a randomized
                algorithm with polynomial expected running time that is
                guaranteed to produce a truth assignment satisfying at least a
                $1-(1/2)^t$ fraction of all clauses.
            \end{lemma}
            \begin{proof}
                We propose the random algorithm that generate random assignment
                (with $1/2$ probability of truth) independently and continuously
                until a truth assignment satisfying at least a $(1-(1/2)^t)$ is
                generated. Next we show that this algorithm has polynomial
                expected running time.
                
                Denote $p_j$ as the probability that a single round of random
                truth assignment results in exactly $j$ clauses being satisfied.
                From Lemma~\ref{lemma:expected_tAT}, we have
                \begin{align}
                    \sum_{j=1}^kjp_j = \left[1 -
                    \left(\frac{1}{2}\right)^t\right]k \label{eq:E_j}
                \end{align}
                Denote the probability that a single round of random
                truth assignment results in at least $1-(1/2)^t$
                fraction of all clauses as $p$:
                \begin{align}
                    p = \sum_{j\geq[1-(1/2)^t]k}p_j \label{eq:p}
                \end{align}
                Define $k' = \lceil (1-(1/2)^t)k\rceil - 1$, then from
                Eq.~(\ref{eq:E_j}) and Eq.~(\ref{eq:p}) we have
                \begin{align}
                    \left[1-\left(\frac{1}{2}\right)^t\right]k & \leq
                    \sum_{j<[1-(1/2)^t]k}k'p_j + \sum_{j\geq[1-(1/2)^t]k}kp_j
                    \notag \\
                    & = k'(1-p) + kp \notag\\
                    & \leq k'+kp
                \end{align}
                Therefore
                \begin{align}
                    p & \geq \left[1-\left(\frac{1}{2}\right)^t\right] -
                    \frac{k'}{k} \notag \\
                    & \geq \frac{1}{k2^t}
                \end{align}
                since both $k$ and $k'$ are integers. Consequently, according to
                Bernoulli distribution, the expected running time of this random
                algorithm is 
                \begin{align}
                    E_{run} = \frac{1}{p} = k2^t
                \end{align}
                which is linear w.r.t the number of clauses.
            \end{proof}
            
        }

    \end{homeworkProblem}
    
    %----------------------------------------------------------------------------------------
    %   PROBLEM 3
    %----------------------------------------------------------------------------------------
    \begin{homeworkProblem}
        Do problem 7 in chapter 13 of the book. How does the answer to this
        problem relate to the answer of problem 2?
        
        In Section 13.4, we designed an approximation algorithm to within a
        factor of $7/8$ for the MAX 3-SAT Problem, where we assumed that each
        clause has terms associated with three different variables. In this
        problem, we will consider the analogous MAX SAT Problem: Given a set of
        clauses $C_1,\ldots,C_k$ over a set of variables $X =
        \{x_1,\ldots,x_n\}$, find a truth assignment satisfying as many of the
        clauses as possible. Each clause has at least one term in it, and all
        the variables in a single clause are distinct, but otherwise we do not
        make any assumptions on the length of the clauses: There may be clauses
        that have a lot of variables, and others may have just a single
        variable.
        
        \begin{homeworkSection}{\homeworkProblemName(a)}
            First consider the randomized approximation algorithm we used for
            MAX 3-SAT, setting each variable independently to \emph{true} or
            \emph{false} with probability $1/2$ each. Show that the expected
            number of clauses satisfied by this random assignment is at least
            $k/2$, that is, at least half of the clauses are satisfied in
            expectation. Give an example to show that there are MAX SAT
            instances such that no assignment satisfies more than half of the
            clauses.
            
            \vspace{10pt}
            \problemAnswer{
                As in Problem 2, denote $Z_i = 1$ if clause $C_i$ is satisfied
                and $Z_i = 0$ otherwise. The expectation of the random variable
                $Z_i$ is
                \begin{align}
                    E[Z_i] & = P\{Z_i = 1\}\cdot 1 + P\{Z_i = 0\}\cdot 0
                    \notag \\
                    & = P\{Z_i = 1\} \notag \\
                    & = 1 - P\{\mbox{All $t_i$ literals in clause $C_i$ are
                    false}\} \notag \\
                    & = 1 - \left(\frac{1}{2}\right)^{t_i} \notag \\
                    & \geq \frac{1}{2}
                \end{align}
                where $t_i \geq 1$ is the number of literals in $C_i$.
                Consequently, the expected number of clauses satisfied by the
                random assignment is
                \begin{align}
                    E[Z] & = \sum_{i=1}^kE[Z_i] \notag\\
                    & \geq \frac{k}{2}
                \end{align}
                
                An example of MAX SAT instances where no assignment satisfies
                half of the clauses is
                \begin{align}
                    C_1 = x_1,\, C_2 = \bar{x_1}
                \end{align}
                apparently any assignment can only satisfy half of the clauses.
            }
        \end{homeworkSection}
        
        \begin{homeworkSection}{\homeworkProblemName(b)}
            If we have a clause that consists only of a single term (e.g., a
            clause consisting just of $x_1$, or just of $\bar{x_2}$), then there
            is only a single way to satisfy it: We need to set the corresponding
            variable in the appropriate way. If we have two clauses such that
            one consists of just the term $x_i$, and the other consists of just
            the negated term $\bar{x_i}$, then this is a pretty direct
            contradiction.
            
            Assume that our instance has no such pair of ``contacting clauses'';
            that is, for no variable $x_i$ do we have both a clause $C=\{x_i\}$
            and a clause $C' = \{\bar{x_i}\}$. Modify the randomized procedure
            above to improve the approximation factor from $1/2$ to at least
            $0.6$. That is, change the algorithm so that the expected number of
            clauses satisfied by the process is at least $0.6k$.
            
            \vspace{10pt}
            \problemAnswer{
                We modify the random algorithm as follows: for all literals that
                appears in a single-literal clause, let the probability of
                setting the literal so that the clause is satisfied to be $0.6$;
                For all other literals, let the probability of setting them to
                \emph{true} to be $0.5$. 
                
                \begin{lemma}
                    This algorithm results in the expected number of satisfied
                    clauses to be $0.6k$
                \end{lemma}
                \begin{proof}
                    Here we use the same notation as in the proof to
                    Lemma~\ref{lemma:expected_tAT}. Consider clause $C_i$
                    \begin{itemize}
                        \item If $C_i$ is a single-literal clause, we have $E[Z_i]
                        = P\{Z_i = 1\} = 0.6$.
                        \item If $C_i$ contains at least two literals, i.e. $t_i
                        \geq 2$, we can divide the literals in this clause into 3
                        classes:
                        \begin{enumerate}
                            \item The literal corresponds to one of the
                            single-literal clause. We assume there are $a_i$ such
                            literals.
                            \item The inverse of the literal corresponds to one of
                            the single-literal clause. We assume there are $b_i$
                            such literals.
                            \item The literal or its inverse don't correspond to
                            any single-literal clause. We assume there are $c_i$
                            such literals.
                        \end{enumerate}
                        Based on this classification, we have $a_i + b_i + c_i = t_i
                        \geq 2$. Then
                        \begin{align}
                            E[Z_i] &= P\{Z_i = 1\} \notag \\
                            & = 1 -
                            \left(\frac{1}{2}\right)^{c_i} \cdot0.4^{a_i}
                            \cdot0.6^{b_i} \notag \\
                            & \geq 1 - 0.6 ^ 2 \notag \\
                            & = 0.64 \notag \\
                            & > 0.6
                        \end{align}
                    \end{itemize}
                    Consequently, $E[Z] = \sum_{i=1}^kE[Z_i] \geq 0.6k$.
                \end{proof}
                
            
            }
        \end{homeworkSection}
        
        \begin{homeworkSection}{\homeworkProblemName(c)}
            Give a randomized polynomial-time algorithm for the general MAX SAT
            Problem, so that the expected number of clauses satisfied by the
            algorithm is at least a $0.6$ fraction of the maximum possible.
            
            (Note that, by the example in part (a), there are instances where
            one cannot satisfy more than $k/2$ clauses; the point here is that
            we'd still like an efficient algorithm that, in expectation, can
            satisfy a $0.6$ fraction of the maximum that can be satisfied by an
            optimal assignment.)
        
            \vspace{10pt}
            \problemAnswer{
                Similar to the proof of Lemma~\ref{lemma:poly_alg}, we propose
                the random algorithm that generate random assignmentas in
                Problem 3(b) independently between different rounds until a
                truth assignment satisfying at least a $0.6k$ clauses are
                generated. To show that this algorithm has polynomial expected
                running time, denote $p_j$ as the probability that a single
                round of random truth assignment results in exactly $j$ clauses
                being satisfied, and $p$ as
                \begin{align}
                    p = \sum_{j\geq0.6k}p_j
                \end{align}
                Also define $k' = \lceil 0.6k\rceil - 1$. From Problem 3(b), we
                have
                \begin{align}
                    0.6k & \leq
                    \sum_{j<0.6k}k'p_j + \sum_{j\geq0.6k}kp_j
                    \notag \\
                    & = k'(1-p) + kp \notag\\
                    & \leq k'+kp
                \end{align}
                Therefore
                \begin{align}
                    p & \geq 0.6 - \frac{k'}{k}
                    \notag \\
                    & \geq \frac{1}{5k}
                \end{align}
                since both $k$ and $k'$ are integers. Consequently, according to
                Bernoulli distribution, the expected running time of this random
                algorithm is 
                \begin{align}
                    E_{run} = \frac{1}{p} = 5k
                \end{align}
                which is linear w.r.t the number of clauses.
                
            }
        \end{homeworkSection}
        
    \end{homeworkProblem}
    %\clearpage
    
    %----------------------------------------------------------------------------------------
    %   PROBLEM 4
    %----------------------------------------------------------------------------------------
    \begin{homeworkProblem}
        Assume that SAT is NP-complete. Define twice-SAT as the problem of
        determing whether a given Boolean formula can be satisfied in at least
        two \emph{different} ways. Two ways to satisfy a Boolean formula are
        different if at least one variable is set differently (i.e., true in one
        and false in the other).
        
        Show how to reduce the SAT problem to the twice-SAT problem in
        polynomial time.
        
        Assuming SAT is NP-complete, show that twice-SAT is NP complete.

        \vspace{10pt}
        \problemAnswer{
                
        }
        
    \end{homeworkProblem}
    %\clearpage
    
    %----------------------------------------------------------------------------------------
    %   PROBLEM 5
    %----------------------------------------------------------------------------------------
    \begin{homeworkProblem}
        \textbf{Approximation Algorithm for Node Cover}
        
        Recall the node cover problem:
        
        Let $G$ be an undirected graph with each node $i$ given weight $w(i) >
        0$. A set of nodes $S$ is a \emph{node cover} of $G$ if every edge of
        $G$ is incident to at least one node of $S$. The \emph{weight} of a node
        cover $S$ is the summation of the weights, denoted $w(S)$, of the nodes
        in $S$; the weighted node cover problem is to select a node cover with
        minimum weight.
        
        The node cover problem (even when all weights are one) is known to be
        NP-hard, and hence we do not expect to find a polynomial-time (in terms
        of worst case) algorithm that is always correct. Therefore, we relax
        somewhat the insistence that the method be both correct and efficient
        for all problem instances. There are many types of relaxations that have
        been developed for NP-hard problems. The most common is the
        constant-factor, polynomial-time approximation algorithm.
        
        For a graph $G$ with node weights, let $S^∗(G)$ denote the minimum
        weight node cover. Let $A$ be a polynomial time algorithm that always
        finds a node cover, but one that is not necessarily minimum; let $S(G)$
        denote the node cover of $G$ that $A$ finds. Then $A$ is called a
        \emph{constant-error polynomial-time approximation algorithm} (or
        approximation algorithm for short) if for any graph $G$,
        $S(G)/S^∗(G)\leq c$ for some fixed constant $c$.
        
        For the node cover problem we will give an approximation algorithm,
        based on network flow, with $c = 2$. First, recall that the node cover
        problem has a nice solution when the graph $G$ is bipartite. This was
        done on a previous homework. You may take it as a black box at this
        point.
        
        \textbf{The Approximation Algorithm for General Graphs}
        
        Given $G$ (not necessarilly bipartite), create bipartite graph $B = (N,
        N', E)$ as follows: for each node $i$ in $G$, create two nodes $i$ and
        $i'$, placing $i$ on the $N$ side, and $i'$ on the $N'$ side of $B$;
        give both of these nodes the weight $w(i)$ of the original node $i$ in
        $G$. If $(i, j)$ is an edge in $G$, create an edge in $B$ from $i$ to
        $j'$ and one from $j$ to $i'$. Now find a minimum cost node cover $S^∗
        (B)$ of graph $B$. From $S^∗(B)$, create a node cover $S(G)$ in $G$ as
        follows: for any node $i$ in $G$, if either $i$ or $i'$ is in $S^∗(B)$,
        then put $i$ in $S(G)$.

        \begin{homeworkSection}{\homeworkProblemName(a)}
            It is easy to find examples where $S(G)$ is not a minimum node cover
            of $G$, and where $S(G)/S^∗ (G) = 2$. Do it.
            
            \vspace{10pt}
            \problemAnswer{
                
            }
        \end{homeworkSection}
        
        \begin{homeworkSection}{\homeworkProblemName(b)}
            However, no worse error ever happens.
            
            \begin{theorem}
                $S(G)/S^∗(G) \leq 2$ for any $G$ and any choice of node weights
                for $G$.
            \end{theorem}
            
            Prove the theorem.
            
            \vspace{10pt}
            \problemAnswer{
                
                
            }
        \end{homeworkSection}
        
    \end{homeworkProblem}
    %\clearpage
    
    %----------------------------------------------------------------------------------------

\end{document}