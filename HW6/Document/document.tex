%%%%%%%%%%%%%%%%%%%%%%%%%%%%%%%%%%%%%%%%%
% Structured General Purpose Assignment
% LaTeX Template
%
% This template has been downloaded from:
% http://www.latextemplates.com
%
% Original author:
% Ted Pavlic (http://www.tedpavlic.com)
%
% Note:
% The \lipsum[#] commands throughout this template generate dummy text
% to fill the template out. These commands should all be removed when 
% writing assignment content.
%
%%%%%%%%%%%%%%%%%%%%%%%%%%%%%%%%%%%%%%%%%

%----------------------------------------------------------------------------------------
%   PACKAGES AND OTHER DOCUMENT CONFIGURATIONS
%----------------------------------------------------------------------------------------

\documentclass{article}

\usepackage{fancyhdr} % Required for custom headers
\usepackage{lastpage} % Required to determine the last page for the footer
\usepackage{extramarks} % Required for headers and footers
\usepackage{graphicx} % Required to insert images
\usepackage{lipsum} % Used for inserting dummy 'Lorem ipsum' text into the template
\usepackage{amsmath}
\usepackage{xcolor}
\usepackage{listings}
\usepackage[toc,page]{appendix}
\usepackage{algorithm}
\usepackage{algorithmic}

% Margins
\topmargin=-0.45in
\evensidemargin=0in
\oddsidemargin=0in
\textwidth=6.5in
\textheight=9.0in
\headsep=0.25in 

\linespread{1.1} % Line spacing

% Set up the header and footer
\pagestyle{fancy}
\lhead{\hmwkAuthorName} % Top left header
\chead{\hmwkClass\ (\hmwkClassInstructor): \hmwkTitle} % Top center header
\rhead{\firstxmark} % Top right header
\lfoot{\lastxmark} % Bottom left footer
\cfoot{} % Bottom center footer
\rfoot{Page\ \thepage\ of\ \pageref{LastPage}} % Bottom right footer
\renewcommand\headrulewidth{0.4pt} % Size of the header rule
\renewcommand\footrulewidth{0.4pt} % Size of the footer rule

\setlength\parindent{0pt} % Removes all indentation from paragraphs

%----------------------------------------------------------------------------------------
%   DOCUMENT STRUCTURE COMMANDS
%   Skip this unless you know what you're doing
%----------------------------------------------------------------------------------------

% Header and footer for when a page split occurs within a problem environment
\newcommand{\enterProblemHeader}[1]{
    \nobreak\extramarks{#1}{#1 continued on next page\ldots}\nobreak
    \nobreak\extramarks{#1 (continued)}{#1 continued on next page\ldots}\nobreak
}

% Header and footer for when a page split occurs between problem environments
\newcommand{\exitProblemHeader}[1]{
    \nobreak\extramarks{#1 (continued)}{#1 continued on next page\ldots}\nobreak
    \nobreak\extramarks{#1}{}\nobreak
}

\setcounter{secnumdepth}{0} % Removes default section numbers
\newcounter{homeworkProblemCounter} % Creates a counter to keep track of the number of problems
\setcounter{homeworkProblemCounter}{0}

\newcommand{\homeworkProblemName}{}
\newenvironment{homeworkProblem}[1][Problem \arabic{homeworkProblemCounter}]{ % Makes a new environment called homeworkProblem which takes 1 argument (custom name) but the default is "Problem #"
    \stepcounter{homeworkProblemCounter} % Increase counter for number of
% problems
    \renewcommand{\homeworkProblemName}{#1} % Assign \homeworkProblemName the
% name of the problem
    \section{\homeworkProblemName} % Make a section in the document with the
% custom problem count
    \enterProblemHeader{\homeworkProblemName} % Header and footer within the
% environment
}{
    \exitProblemHeader{\homeworkProblemName} % Header and footer after the
% environment
}

\newcommand{\problemAnswer}[1]{ % Defines the problem answer command with the content as the only argument
    \noindent\textbf{\emph{Answer: }}#1 % Just put a keyword Answer in
    % bold/italic at the beginning
}

\newcommand{\homeworkSectionName}{}
\newenvironment{homeworkSection}[1]{ % New environment for sections within homework problems, takes 1 argument - the name of the section
    \renewcommand{\homeworkSectionName}{#1} % Assign \homeworkSectionName to the
% name of the section from the environment argument
    \subsection{\homeworkSectionName} % Make a subsection with the custom name
% of the subsection
    \enterProblemHeader{\homeworkProblemName\ [\homeworkSectionName]} % Header
% and footer within the environment
}{
    \enterProblemHeader{\homeworkProblemName} % Header and footer after the
% environment
}

\newtheorem{theorem}{Theorem}[homeworkProblemCounter]
\newtheorem{lemma}[theorem]{Lemma}
\newtheorem{proposition}[theorem]{Proposition}
\newtheorem{corollary}[theorem]{Corollary}

\newenvironment{proof}[1][Proof]{
    \begin{trivlist}
        \item[\hskip \labelsep {\bfseries #1}]
    }{
        \end{trivlist}
}
\newenvironment{definition}[1][Definition]{
    \begin{trivlist}
        \item[\hskip \labelsep {\bfseries #1}]
    }{
        \end{trivlist}
}
\newenvironment{example}[1][Example]{
    \begin{trivlist}
        \item[\hskip \labelsep {\bfseries #1}]
    }{
        \end{trivlist}
    }
\newenvironment{remark}[1][Remark]{
    \begin{trivlist}
        \item[\hskip \labelsep {\bfseries #1}]
    }{
        \end{trivlist}
}

\newcommand{\qed}{
    \nobreak \ifvmode \relax \else
    \ifdim\lastskip<1.5em \hskip-\lastskip
    \hskip1.5em plus0em minus0.5em \fi \nobreak
    \vrule height0.75em width0.5em depth0.25em\fi
}

\lstset{
    frame=single,
    breaklines=true,
    postbreak=\raisebox{0ex}[0ex][0ex]{\ensuremath{\color{red}\hookrightarrow\space}}
}
   
%----------------------------------------------------------------------------------------
%   NAME AND CLASS SECTION
%----------------------------------------------------------------------------------------

\newcommand{\hmwkTitle}{Assignment\ \#6} % Assignment title
\newcommand{\hmwkDueDate}{Thursday,March\ 5,\ 2015} % Due date
\newcommand{\hmwkClass}{ECS\ 222A} % Course/class
\newcommand{\hmwkClassTime}{TR 4:40pm-6:00pm} % Class/lecture time
\newcommand{\hmwkClassInstructor}{Daniel Gusfield} % Teacher/lecturer
\newcommand{\hmwkAuthorName}{Wenhao Wu} % Your name

%----------------------------------------------------------------------------------------
%   TITLE PAGE
%----------------------------------------------------------------------------------------

\title{
    \vspace{2in}
    \textmd{\textbf{\hmwkClass:\ \hmwkTitle}}\\
    \normalsize\vspace{0.1in}\small{Due\ on\ \hmwkDueDate}\\
    \vspace{0.1in}\large{\textit{\hmwkClassInstructor\ \hmwkClassTime}}
    \vspace{3in}
}

\author{\textbf{\hmwkAuthorName}}
\date{} % Insert date here if you want it to appear below your name

%----------------------------------------------------------------------------------------

\begin{document}

    \maketitle
    
    %----------------------------------------------------------------------------------------
    %   TABLE OF CONTENTS
    %----------------------------------------------------------------------------------------
    
    %\setcounter{tocdepth}{1} % Uncomment this line if you don't want subsections listed in the ToC
    
    \newpage
    \tableofcontents
    \newpage

    %----------------------------------------------------------------------------------------
    %   PROBLEM 1
    %----------------------------------------------------------------------------------------
    \begin{homeworkProblem}
        We saw a randomized algorithm that tries to find a global $Min$ cut
        in an undirected, unweighted graph $G$ with $m$ edges. Now suppose we want
        to find a cut that has a \emph{large} number of edges, i.e., a partition
        of the nodes of $G$ into two sets $S$ and $T$ so that the number of
        edges that have one node in $S$ and one node in $T$ is large. Denote the
        maximum possible number as $Max(G)$. The problem of finding $Max(G)$ is
        NP-hard, so we would like a randomized algorithm that finds an $S$, $T$
        cut where the \emph{expected} number of edges that cross the cut is a
        large fraction of $Max(G)$.
        
        Here is a particularly brainless algorithm that does it. For each
        vertex, flip a coin: if the coin comes up heads, put the vertex in $S$,
        otherwise, put it in $T$. Assume that the coin is fair, i.e., the
        probability of heads is $1/2$.
        
        \begin{homeworkSection}{\homeworkProblemName(a)}
            Using this randomized algorithm, what is the expected number of
            edges that have one node in $S$ and one in $T$? Explain.
            
            \vspace{10pt}
            \problemAnswer{
            
            }
        \end{homeworkSection}
        
        \begin{homeworkSection}{\homeworkProblemName(b)}
            Prove that in any undirected graph $G$, there is a cut that contains
            at least half of the edges in $G$.
            
            \vspace{10pt}
            \problemAnswer{
            
            }
        \end{homeworkSection}
        
        \begin{homeworkSection}{\homeworkProblemName(c)}
            Often in the analysis of social media, people build graphs
            representing who knows or likes (or hates or dates) whomever else.
            Then they analyze the graphs for particular features, such as large
            cliques, large independent sets, small cuts, nodes with high degree,
            the number of nodes of degree 1, number of triangles, etc. and
            they ascribe a ``meaning'' to each of these features. For example, a
            node with high degree is a ``hub'' or ``kingpin''; and a large
            clique represents a ``socially cohesive unit''; and a small cut is a
            ``bottleneck'', etc. This same approach to studying interactions is
            used in a huge variety of other systems. For example, graphs (call
            them ``biological networks'') are also to represent interactions
            between molecules, or between animals in a biological system, and
            biological, behavioural or chemical meanings are ascribed to
            features in these graphs. In general, such networks are called
            ``interaction networks''.
            
            Now consider a ``large cut'' as a feature of an interaction network.
            Make up (use your imagination) as many meanings you can for a large
            cut in a social or biological network, or any other interaction
            network you can describe.
            
            Given the fact stated in problem 1b, how large must a cut be before
            it could plausibly say anything meaningful about the interactions
            represented in an interaction graph? If you were a ``network
            analyst'' trying to find meaningful information from an interaction
            network, and you didn't know the fact stated in problem 1b, would
            that be a problem?
            
            \vspace{10pt}
            \problemAnswer{
            
            }
        \end{homeworkSection}
    \end{homeworkProblem}
    
    %----------------------------------------------------------------------------------------
    %   PROBLEM 2
    %----------------------------------------------------------------------------------------
    \begin{homeworkProblem}
        Extend the analysis done for 3-SAT (in class and also in Section 13.4)
        to 4-SAT, i.e., the assumption that every clause has 4 literals. Then
        generalize to t-SAT for any fixed integer $t$. That is, generalize
        statments 13.14, 13.15, and 13.16, and justify your answers.
            
        \vspace{10pt}
        \problemAnswer{
                
        }

    \end{homeworkProblem}
    
    %----------------------------------------------------------------------------------------
    %   PROBLEM 3
    %----------------------------------------------------------------------------------------
    \begin{homeworkProblem}
        Do problem 7 in chapter 13 of the book. How does the answer to this
        problem relate to the answer of problem 2?
        
        \vspace{10pt}
        \problemAnswer{
            
        }
    \end{homeworkProblem}
    %\clearpage
    
    %----------------------------------------------------------------------------------------
    %   PROBLEM 4
    %----------------------------------------------------------------------------------------
    \begin{homeworkProblem}
        Assume that SAT is NP-complete. Define twice-SAT as the problem of
        determing whether a given Boolean formula can be satisfied in at least
        two \emph{different} ways. Two ways to satisfy a Boolean formula are
        different if at least one variable is set differently (i.e., true in one
        and false in the other).
        
        Show how to reduce the SAT problem to the twice-SAT problem in
        polynomial time.
        
        Assuming SAT is NP-complete, show that twice-SAT is NP complete.

        \vspace{10pt}
        \problemAnswer{
                
        }
        
    \end{homeworkProblem}
    %\clearpage
    
    %----------------------------------------------------------------------------------------
    %   PROBLEM 5
    %----------------------------------------------------------------------------------------
    \begin{homeworkProblem}
        \textbf{Approximation Algorithm for Node Cover}
        
        Recall the node cover problem:
        
        Let $G$ be an undirected graph with each node $i$ given weight $w(i) >
        0$. A set of nodes $S$ is a \emph{node cover} of $G$ if every edge of
        $G$ is incident to at least one node of $S$. The \emph{weight} of a node
        cover $S$ is the summation of the weights, denoted $w(S)$, of the nodes
        in $S$; the weighted node cover problem is to select a node cover with
        minimum weight.
        
        The node cover problem (even when all weights are one) is known to be
        NP-hard, and hence we do not expect to find a polynomial-time (in terms
        of worst case) algorithm that is always correct. Therefore, we relax
        somewhat the insistence that the method be both correct and efficient
        for all problem instances. There are many types of relaxations that have
        been developed for NP-hard problems. The most common is the
        constant-factor, polynomial-time approximation algorithm.
        
        For a graph $G$ with node weights, let $S^∗(G)$ denote the minimum
        weight node cover. Let $A$ be a polynomial time algorithm that always
        finds a node cover, but one that is not necessarily minimum; let $S(G)$
        denote the node cover of $G$ that $A$ finds. Then $A$ is called a
        \emph{constant-error polynomial-time approximation algorithm} (or
        approximation algorithm for short) if for any graph $G$,
        $S(G)/S^∗(G)\leq c$ for some fixed constant $c$.
        
        For the node cover problem we will give an approximation algorithm,
        based on network flow, with $c = 2$. First, recall that the node cover
        problem has a nice solution when the graph $G$ is bipartite. This was
        done on a previous homework. You may take it as a black box at this
        point.
        
        \textbf{The Approximation Algorithm for General Graphs}
        
        Given $G$ (not necessarilly bipartite), create bipartite graph $B = (N,
        N', E)$ as follows: for each node $i$ in $G$, create two nodes $i$ and
        $i'$, placing $i$ on the $N$ side, and $i'$ on the $N'$ side of $B$;
        give both of these nodes the weight $w(i)$ of the original node $i$ in
        $G$. If $(i, j)$ is an edge in $G$, create an edge in $B$ from $i$ to
        $j'$ and one from $j$ to $i'$. Now find a minimum cost node cover $S^∗
        (B)$ of graph $B$. From $S^∗(B)$, create a node cover $S(G)$ in $G$ as
        follows: for any node $i$ in $G$, if either $i$ or $i'$ is in $S^∗(B)$,
        then put $i$ in $S(G)$.

        \begin{homeworkSection}{\homeworkProblemName(a)}
            It is easy to find examples where $S(G)$ is not a minimum node cover
            of $G$, and where $S(G)/S^∗ (G) = 2$. Do it.
            
            \vspace{10pt}
            \problemAnswer{
                
            }
        \end{homeworkSection}
        
        \begin{homeworkSection}{\homeworkProblemName(b)}
            However, no worse error ever happens.
            
            \begin{theorem}
                $S(G)/S^∗(G) \leq 2$ for any $G$ and any choice of node weights
                for $G$.
            \end{theorem}
            
            Prove the theorem.
            
            \vspace{10pt}
            \problemAnswer{
                
                
            }
        \end{homeworkSection}
        
    \end{homeworkProblem}
    %\clearpage
    
    %----------------------------------------------------------------------------------------

\end{document}